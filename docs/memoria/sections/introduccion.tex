\documentclass[../main.tex]{subfiles}

\begin{document}

\subsection{Motivación}
La motivación principal para realizar este proyecto es el potencial que ofrece mezclar dos campos que últimamente están creciendo exponencialmente: la \textbf{robótica social} y la \textbf{inteligencia artificial}. Gracias 
a juntar ambos conceptos, conseguimos que el robot sea capaz de interactuar de forma más humana, respondiendo frases más allá de un conjunto de respuestas programadas, y comprendiendo
de manera más inteligente qué es lo que le está queriendo transmitir la persona. Además de lo mencionado, utilizaré también una estructura que se llama \textbf{árboles de comportamiento}, utilizado principalmente
en robots o en PNJ (Personaje No Jugador) en los videojuegos, para planificar de forma inteligente las acciones en el trascurso de su realización.

\subsection{Objetivos}
Los objetivos principales del proyecto son los siguientes:
\begin{itemize}
    
    \item Implementar y adaptar las tareas que pueda realizar el robot para que sean planificables por el árbol de comportamiento
\end{itemize}

\subsection{Estructura del documento}
\blindtext

\subsection{Tecnologías utilizadas}

\subsubsection{ROS2}
ROS2 es un \textit{framework} diseñado para el desarrollo de software para robots, el cual proporciona servicios como control de hardware,
abstracción, gestión de dispositivos y comunicación entre procesos (o nodos) y paquetes del sistema. Aunque pudiera parecerlo, ROS2 no es un sistema operativo, sino 
que es un conjunto de herramientas y bibliotecas cuya finalidad principal consiste en facilitar el desarrollo de aplicaciones robóticas complejas,
y su escalabilidad.

\subsubsection{Python}
Python es un lenguaje de programación multiparadigma y multiplataforma. Este lenguaje es muy simple y fácil de aprender, ya que su 
sintaxis es simple y similar al inglés, lo cual facilita su aprendizaje, y además tiene un tipado dinámico que facilita la asignación de variables. Además de 
lo mencionado, Python cuenta con un extenso repositorio de librerías, lo cual hace que utilizandolo puedas realizar aplicaciones de distintos tipos (análisis de datos, inteligencia artificial o el
manejo de ROS2).

\subsubsection{Qt}
Qt es un \textit{framework} de desarollo de interfaces gráficas multiplataforma el cúal se puede utilizar tanto con el lenguage de programación \textbf{Python} (mencionado anteriormente) o
con el lenguaje de programación \textbf{C++}.

\subsubsection{Computación en la nube}
Debido a la gran carga computacional que supone un LLM, el propio robot no podría cargar con un servidor lo suficientemente potente para poder tener un modelo local (debido a limitaciones de potencia eléctrica y peso), por lo que se utiliza un computador externo potente (Ultra Edge) que recibe la información del robot mediante los ya mencionados Servicios de ROS2, y tras realizar el cómputo necesario, devuelve la respuesta al robot.

Gracias a esta tecnología, conseguimos generar y transmitir la información necesaria para el robot sin necesidad de afectar al rendimiento del equipo móvil del robot.

\end{document}