\documentclass[../main.tex]{subfiles}

\begin{document}

\subsection{Motivación}
La motivación principal para realizar este proyecto es el potencial que ofrece mezclar dos campos que últimamente están creciendo exponencialmente: la \textbf{robótica social} y la \textbf{inteligencia artificial}. Gracias 
a juntar ambos conceptos, conseguimos que el robot sea capaz de interactuar de forma más humana, respondiendo frases más allá de un conjunto de respuestas programadas, y comprendiendo
de manera más inteligente qué es lo que le está queriendo transmitir la persona. Además de lo mencionado, utilizaré también una estructura que se llama \textbf{árboles de comportamiento}, utilizado principalmente
en robots o en PNJ (Personaje No Jugador) en los videojuegos, para planificar de forma inteligente las acciones en el trascurso de su realización.

\subsection{Objetivos}

El objetivo principal de este TFG es el desarrollo de un sistema capaz de transformar una orden dada en lenguaje natural por el usuario, a un conjunto
de tareas que el robot pueda realizar. La ejecución y planificación de estas tareas se realizará mediante un árbol de comportamiento, el cual decide el orden de las tareas
y se encarga del control de errores de las mismas. A continuación, listaré de forma breve otros objetivos más específicos.

\begin{itemize}
    \item Aprendizaje de características del LLM y obtención de resultados para utilizar el LLM óptimo y más potente
    \item Aprendizaje de las estructuras de información compatibles con los árboles de comportamiento (Behavior Tree)
    
    \item Implementación del puente de comunicación entre el LLM y el \textit{framework} ROS2
    \item Desarrollo de integración de mapas semánticos para ampliar el conocimiento del modelo sobre el entorno actual en el que se encuentra el robot
    \item Desarrollo de una interfaz de monitorización de las tareas, desde la cual podamos enviar la sentencia y seguir el estado de las tareas.
\end{itemize}

\subsection{Estructura del documento}
A continuación, se exponen las distintas secciones en las cuales está dividido este documento:

\begin{enumerate}
    \item \textbf{Introducción}: En esta sección se describe el contexto y conceptos necesarios, seguido de las motivaciones de realización de este TFG y los objetivos
    y resultados de este trabajo. Finalmente, se describen de manera breve las tecnologías utilizadas en este TFG, tanto conceptuales como técnicas.

    \item \textbf{Estado del arte}: En esta sección se abordan el estado del arte de los campos de estudio más relevantes para este trabajo, que en este caso se tratan de la 
    \textbf{robótica social}, \textbf{modelos de lenguaje a gran escala} y \textbf{árboles de comportamiento}. Para cada campo existe una descripción del mismo, y además una explicación del estado
    actual del mismo.

    \item \textbf{Diseño del sistema}: En esta sección se describe el diseño del sistema y se muestran los diagramas realizados durante el proceso. Además, comentaremos los distintos patrones de diseño utilizados
    en el desarrollo del sistema y se explicará la metodología de trabajo utilizada en el trabajo.

    \item \textbf{Desarrollo del sistema}: En esta sección se muestra y se describe todo el desarrollo del resultado final, profundizando sobre cómo está hecha la herramienta y sobre algunas
    interfaces gráficas que se han desarrollado con el objetivo de visualizar los procesos.

    \item \textbf{Análisis y pruebas}: En la sección de análisis y pruebas se exponen los análisis experimentales que se ha realizado para la toma de decisiones del proyecto.

    \item \textbf{Conclusiones y líneas futuras}: Para finalizar, en esta sección se incluye la conclusión personal tras realizar el trabajo y las posibles líneas futuras con las que
    se podría continuar el proyecto.
\end{enumerate}

\subsection{Tecnologías utilizadas}
En esta sección, se describirán las tencologías utilizadas para la realización del proyecto, incluyendo tanto tecnologías específicas como técnicas o estándares:

\subsubsection{ROS2}
\textbf{ROS2} es un \textit{framework} diseñado para el desarrollo de software para robots, el cual proporciona servicios como control de hardware,
abstracción, gestión de dispositivos y comunicación entre procesos (o nodos) y paquetes del sistema. Aunque pudiera parecerlo, ROS2 no es un sistema operativo, sino 
que es un conjunto de herramientas y bibliotecas cuya finalidad principal consiste en facilitar el desarrollo de aplicaciones robóticas complejas,
y su escalabilidad.

\subsubsection{Python}
\textbf{Python} es un lenguaje de programación multiparadigma y multiplataforma. Este lenguaje es muy simple y fácil de aprender, ya que su 
sintaxis es simple y similar al inglés, lo cual facilita su aprendizaje, y además tiene un tipado dinámico que facilita la asignación de variables. Además de 
lo mencionado, Python cuenta con un extenso repositorio de librerías, lo cual hace que utilizandolo puedas realizar aplicaciones de distintos tipos (análisis de datos, inteligencia artificial o el
manejo de ROS2).

\subsubsection{Qt}
\textbf{Qt} es un \textit{framework} de desarollo de interfaces gráficas multiplataforma el cúal se puede utilizar tanto con el lenguage de programación \textbf{Python} (mencionado anteriormente) o
con el lenguaje de programación \textbf{C++}.

\subsubsection{Computación en la nube}
Debido a la gran carga computacional que supone un LLM, el propio robot no podría cargar con un servidor lo suficientemente potente para poder tener un modelo local (debido a limitaciones de potencia eléctrica y peso), por lo que se utiliza un computador externo potente (Ultra Edge) que recibe la información del robot mediante los ya mencionados Servicios de ROS2, y tras realizar el cómputo necesario, devuelve la respuesta al robot.

Gracias a esta tecnología, conseguimos generar y transmitir la información necesaria para el robot sin necesidad de afectar al rendimiento del equipo móvil del robot.

\subsubsection{Py Trees}
\textbf{Py Trees} es una librería para el lenguaje de programación Python la cual es una implementación de los árboles de comportamiento en \textit{Python}. Esta librería
incluye tanto clases para definir comportamientos dentro de un árbol, como la implementación de la lógica de un árbol de comportamiento y su ejecución.

\end{document}