\documentclass[../main.tex]{subfiles}

\begin{document}
%% begin abstract format
\makeatletter
\renewenvironment{abstract}{%
    \if@twocolumn
      \section*{Resumen \\}%
    \else %% <- here I've removed \small
    \begin{flushright}
        {\filleft\Huge\bfseries\fontsize{48pt}{12}\selectfont Resumen\vspace{\z@}}%  %% <- here I've added the format
        \end{flushright}
      \quotation
    \fi}
    {\if@twocolumn\else\endquotation\fi}
\makeatother
%% end abstract format
%% begin abstract format
\makeatletter
\renewenvironment{abstract}{%
    \if@twocolumn
      \section*{Resumen \\}%
    \else %% <- here I've removed \small
    \begin{flushright}
        {\filleft\Huge\bfseries\fontsize{48pt}{12}\selectfont Resumen\vspace{\z@}}%  %% <- here I've added the format
        \end{flushright}
      \quotation
    \fi}
    {\if@twocolumn\else\endquotation\fi}
\makeatother
%% end abstract format
\begin{abstract}

A día de hoy, conceptos como la \textit{robótica social}~\cite{canete2024multimodal} van tomando más importancia, debido a que la interacción de los humanos con los robots es más frecuente en distintos ámbitos de la sociedad, ya sea una llamada de soporte de cualquier empresa o para recibir comida de un restaurante. Aparte de la robótica social, otro campo que también ha evolucionado, aunque este ha sido a nivel exponencial, son los \textit{Modelos de Lenguaje a Gran Escala} (LLM), que han pasado en apenas unos años de generar respuestas con sentido, pero con algunos fallos de razonamiento, a poder generar respuestas complejas, tal y como las razonaría un ser humano, y no tan solo respuestas textuales, sino que tienen la capacidad de realizar imágenes realistas, vídeos y música.

Este Trabajo de Fin de Grado pretende contribuir a la robótica social, mediante la combinación de dicho campo con las LLM, obteniendo una interacción, por parte del robot, más humana y con un razonamiento que mejore la experiencia. En concreto, el trabajo se aplicará sobre un robot móvil, enfocándose principalmente en la identificación y planificación de  tareas, las cuales se identificarán mediante una o varias ordenes dadas por un agente humano de forma oral o textual. En cuanto a la planificación de la ejecución de las tareas, se planificará utilizando la lógica de los \textit{árboles de comportamiento}~\cite{ogren2022behavior}.

\bfseries{\large{Palabras clave:}} Robótica Social, Inteligencia Artificial, LLM, Robot Móvil, Árboles de comportamiento

\end{abstract}
\end{document}


