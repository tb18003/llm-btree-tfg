\documentclass[../main.tex]{subfiles}

\begin{document}
%% begin abstract format
\makeatletter
\renewenvironment{abstract}{%
    \if@twocolumn
      \section*{Abstract \\}%
    \else %% <- here I've removed \small
    \begin{flushright}
        {\filleft\Huge\bfseries\fontsize{48pt}{12}\selectfont Abstract\vspace{\z@}}%  %% <- here I've added the format
        \end{flushright}
      \quotation
    \fi}
    {\if@twocolumn\else\endquotation\fi}
\makeatother
%% end abstract format
\begin{abstract}

In these days, meanings like \textit{social robotics}~\cite{canete2024multimodal} get more significance, due to the interaction of humans with robots is more frequent in distinct fields of society, such as a support call from a company or receiving food from a restaurant. Besides social robotics, another field that has evolved exponentially is the \textit{Large Language Models} (LLM), which in a few years, were capable of generating answers that make sense, but fail sometimes, and now they are capable of generate complex answers, reasoning like a human, and generating realistic pictures, videos and music.

This project wants to contribute to social robotics by combining this field with LLMs, obtaining a more human robot interaction, adding a better reasoning and better experience for the user. Specifically, the project will be applied in a mobile robot, focusing firstly in the identification and planification of tasks, which will be identified by one or more orders given by an human agent in oral or textual way. The planification of tasks execution, will be scheduled using the \textit{behaviour trees}~\cite{ogren2022behavior} logic.

\bfseries{\large{Keywords:}} Social Robotics, Artificial Intelligence, LLM, Mobile Robot, Behaviour Tree

\end{abstract}
\end{document}